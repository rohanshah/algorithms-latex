\documentclass[12pt]{article}
\usepackage{fullpage}
\usepackage{titlesec}
\usepackage{tikz}
\usepackage{amsfonts,amssymb}
\usepackage{amsmath}
\usepackage{comment}
\relpenalty=9999
\binoppenalty=9999

\begin{document}
\pagestyle{plain}
\titleformat{\subsection}[runin]
  {\normalfont\large\bfseries}{\thesubsection}{1em}{}
\titleformat{\subsubsection}[runin]
  {\normalfont\large\bfseries}{\thesubsubsection}{1em}{}

\section*{Problem 5a.}
Consider the following algorithm for computing $\lambda(v)$ for all vertices in
$V$. Sort all the $r(u)$ values from least to greatest. Let $R_G = (V, R_E)$ be
the reverse graph of $G$ such that $R_E = \{(v,u) \mid (u,v) \in E \}$. Perform
a depth-first search from $u$ in $R_G$ where $u$ is the vertex with the least
$r(u)$ value. For every vertex $v$ reachable from $u$ we set $\lambda(v) = r(u)$
and increment a counter on $v$ which is initialized at $0$. This process is then
repeated using the next smallest $r(u')$. When a vertex $v$'s counter reaches
$t$ we output its $\lambda(v)$ and break all incoming edges to $v$ in $R_G$.
Clearly, once a vertex's counter reaches $t$ we have found its $\lambda(v)$
since we've seen the first $t$ smallest $r(u)$ values in $S(v)$. In this
algorithm every vertex is seen at most $t$ times and every edge is traveresed at
most $t$ times since it is broken after its $t$'th traversal. So the algorithm
runs in $O(m\cdot t + n\cdot t) = O((m +n)t)$ time.

\section*{Problem 5b.}
Consider the two inequalites seperately. The first inequality
$$ (1-\epsilon)|S(v)| \le \frac{n\cdot t}{\lambda(v)} $$
can be rewritten as
$$ \lambda(v) \le \frac{n\cdot t}{(1-\epsilon)|S(v)|} $$
Since $\lambda(v)$ represents the $t$'th smallest $r(u)$, where $u \in S(v)$,
for the inequality to hold, there must be at least $t$ such
$r(u)$'s less than or equal $\frac{n\cdot t}{(1-\epsilon)|S(v)|}$. Let
$X = X_1 + \cdots + X_{|S(v)|}$ be the sum of binary independent random variables
such that $X_i = 1$ if $r(u_i) \le \frac{n\cdot t}{(1-\epsilon)|S(v)|}$ where
$u_i$ is an element of $S(v)$. Since each $v \in V$ is assigned a value $r(v)$
from $\{1...n\}$ uniformly at random, $X_i$ is a binary independent random
variable. Therefore
$$\textbf{E}[X_i] = \textbf{Pr}[X_i = 1] =
\frac{\frac{n\cdot t}{(1-\epsilon)|S(v)|}}{n} =
\frac{t}{(1-\epsilon)|S(v)|}$$
By the linearity of expectation we have
$$\textbf{E}[X] = \frac{t}{(1-\epsilon)|S(v)|}\cdot |S(v)| =
\frac{t}{(1-\epsilon)}$$
From above, we need $X$ to be greater than or equal to $t$ for success. We can
bound the probability of failure using Chernoff Bounds
$$\textbf{Pr}[X < t] = \textbf{Pr}[X < (1-\epsilon)\cdot\frac{t}{(1-\epsilon)}]=
\textbf{Pr}[X < (1-\epsilon)\cdot\textbf{E}[X]] <
\frac{1}{n^{\frac{18}{(1-\epsilon)}}}$$
So with probability $1-1/\text{poly}(n)$ the first inequality holds.
The second inequality
$$ (1+\epsilon)|S(v)| \ge \frac{n\cdot t}{\lambda(v)} $$
can be rewritten as
$$ \lambda(v) \ge \frac{n\cdot t}{(1+\epsilon)|S(v)|} $$
Clearly the inequality fails when there are more than $t$ such
$r(u)$'s less than $\frac{n\cdot t}{(1+\epsilon)|S(v)|}$. Let
$X = X_1 + \cdots + X_{|S(v)|}$ be the sum of binary independent random variables
such that $X_i = 1$ if $r(u_i) < \frac{n\cdot t}{(1+\epsilon)|S(v)|}$ where
$u_i$ is an element of $S(v)$. Since each $v \in V$ is assigned a value $r(v)$
from $\{1...n\}$ uniformly at random, $X_i$ is a binary independent random
variable. Therefore
$$\textbf{E}[X_i] = \textbf{Pr}[X_i = 1] =
\frac{\frac{n\cdot t}{(1+\epsilon)|S(v)|}}{n} =
\frac{t}{(1+\epsilon)|S(v)|}$$
By the linearity of expectation we have
$$\textbf{E}[X] = \frac{t}{(1+\epsilon)|S(v)|}\cdot |S(v)| =
\frac{t}{(1+\epsilon)}$$
From above, we need $X$ to be greater than $t$ for the inequality to fail. We
can again bound the probability of failure using Chernoff Bounds
$$\textbf{Pr}[X > t] = \textbf{Pr}[X > (1+\epsilon)\cdot\frac{t}{(1+\epsilon)}]=
\textbf{Pr}[X > (1+\epsilon)\cdot\textbf{E}[X]] <
\frac{1}{n^{\frac{9}{(1+\epsilon)}}}$$
So with probability $1-1/\text{poly}(n)$ the second inequality also holds.
So with probability $1-1/\text{poly}(n)$ the algorithm returns a
$(1 \pm \epsilon)$ estimate of $S(v)$.

\end{document}
