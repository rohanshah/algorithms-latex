\documentclass[12pt]{article}
\usepackage{fullpage}
\usepackage{titlesec}
\usepackage{tikz}
\usepackage{amsfonts,amssymb}
\usepackage{amsmath}
\usepackage{comment}
\relpenalty=9999
\binoppenalty=9999

\begin{document}
\pagestyle{plain}
\titleformat{\subsection}[runin]
  {\normalfont\large\bfseries}{\thesubsection}{1em}{}
\titleformat{\subsubsection}[runin]
  {\normalfont\large\bfseries}{\thesubsubsection}{1em}{}

\section*{Problem 5a.}

\section*{Problem 5b.}
Consider the two inequalites seperately. The first inequality
$$ (1-\epsilon)|S(v)| \le \frac{n\cdot t}{\lambda(v)} $$
can be rewritten as
$$ \lambda(v) \le \frac{n\cdot t}{(1-\epsilon)|S(v)|} $$
Since $\lambda(v)$ represents the $t$'th smallest $r(u)$, where $u \in S(v)$,
for the inequality to hold, there must be at least $t$ such
$r(u)$'s less than or equal $\frac{n\cdot t}{(1-\epsilon)|S(v)|}$. Let
$X = X_1 + \cdots + X_{|S(v)|}$ be the sum of binary independent random variables
such that $X_i = 1$ if $r(u_i) \le \frac{n\cdot t}{(1-\epsilon)|S(v)|}$ where
$u_i$ is an element of $S(v)$. Since each $v \in V$ is assigned a value $r(v)$
from $\{1...n\}$ uniformly at random, $X_i$ is a binary independent random
variable. Therefore
$$\textbf{E}[X_i] = \textbf{Pr}[X_i = 1] =
\frac{\frac{n\cdot t}{(1-\epsilon)|S(v)|}}{n} =
\frac{t}{(1-\epsilon)|S(v)|}$$
By the linearity of expectation we have
$$\textbf{E}[X] = \frac{t}{(1-\epsilon)|S(v)|}\cdot |S(v)| =
\frac{t}{(1-\epsilon)}$$
From above, we need $X$ to be greater than or equal to $t$ for success. We can
bound the probability of failure using Chernoff Bounds
$$\textbf{Pr}[X < t] = \textbf{Pr}[X < (1-\epsilon)\cdot\frac{t}{(1-\epsilon)}]=
\textbf{Pr}[X < (1-\epsilon)\cdot\textbf{E}[X]] <
\frac{1}{n^{\frac{18}{(1-\epsilon)}}}$$
So with probability $1-1/\text{poly}(n)$ the first inequality holds. The second
inequality
$$ (1+\epsilon)|S(v)| \ge \frac{n\cdot t}{\lambda(v)} $$
is symmetrical and the probability of failure can similarly be bounded using
Chernoff Bounds to be $1-1/\text{poly}(n)$. So with probability
$1-1/\text{poly}(n)$ the algorithm returns a $(1 \pm \epsilon)$ estimate of
$S(v)$.

\end{document}
