\documentclass[12pt]{article}
\usepackage{fullpage}
\usepackage{titlesec}
\usepackage{tikz}
\usepackage{amsfonts,amssymb}
\usepackage{amsmath}
\usepackage{comment}
\relpenalty=9999
\binoppenalty=9999

\begin{document}
\pagestyle{plain}
\titleformat{\subsection}[runin]
  {\normalfont\large\bfseries}{\thesubsection}{1em}{}
\titleformat{\subsubsection}[runin]
  {\normalfont\large\bfseries}{\thesubsubsection}{1em}{}

\section*{Problem 3.}
Consider a satisfying 3-coloring assignment $A$ of the graph $G$ using the
colors red, blue, and green. Every triangle in $G$ must have all its vertices
colored differently by the definition of a 3-coloring. Now consider a relaxed
2-coloring assignment $A'$ where every green vertex in $A$ is arbitrarily
colored either red or blue. $A'$ satisfies the constraint that no triangle is
monocromatic. Every triangle contains a red, blue, and green vertice in $A$
therefore every triangle contains at least one red and one blue triangle in $A'$.
Fix such an assignment $A'$ and consider the following algorithm. Arbitrarily
assign a 2-coloring to the graph $G$. While there remain monocromatic triangles
in $G$, arbitrarily select one, randomly choose one of
its vertices and flip its color. We repeat until every vertice in a triangle has
the color assigned to it in $A'$. We know from above that each triangle in $G$
has one vertex who's color must be blue, one vertex who's color must be red and
one vertice who's color does not matter (i.e. the green vertex). Therefore, the
probability that we choose a correctly colored vertice and flip its color is
$\frac{1}{3}$. The probability that we choose an incorrectly colored vertice and
flip its color is $\frac{1}{3}$. And the probability that we choose a vertice
who's coloring does not affect the monocromatic triangle constraint is $\frac{1}{3}$.
We can model this algorithm as the random walk of a path graph defined in the
following way: each vertex $v_i$ has an edge to vertex $v_{i-1}$, $v_{i+1}$,
and a self loop each with equal probability of being traversed. Being at a
vertice $v_i$ corresponds to $i$ vertices in $G$ being correctly colored
according to our satisfying assignment $A'$. We have a path of length at most
$n$ since we only need to consider the vertices who are members of a triangle,
which is at most all the vertices in $G$. The cover time of our path graph is at
most $2\cdot(3n)\cdot(n-1) = 6n^2+6n$ steps, therefore our algorithm takes
$O(n^2)$ expected running time.

\end{document}
