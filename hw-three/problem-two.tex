\documentclass[12pt]{article}
\usepackage{fullpage}
\usepackage{titlesec}
\usepackage{tikz}
\usepackage{amsfonts,amssymb}
\usepackage{amsmath}
\usepackage{comment}
\relpenalty=9999
\binoppenalty=9999

\begin{document}
\pagestyle{plain}
\titleformat{\subsection}[runin]
  {\normalfont\large\bfseries}{\thesubsection}{1em}{}
\titleformat{\subsubsection}[runin]
  {\normalfont\large\bfseries}{\thesubsubsection}{1em}{}

\section*{Problem 2.}
Consider the following strategy for the hunter to catch the rabbit in
expected polynomial time. The hunter chooses a vertex $v$ uniformly at random,
takes the shortest path to $v$ of length $l$ and waits at $v$ for $d-l$ time
steps where $d$ is the diameter of $G$. By waiting $d$ time steps the rabbit
can have no knowledge of the hunter's position since the hunter could be at any
node after $d$ time. The probability that the hunter chose the vertex that the
rabbit would be at at time $d$ is $\frac{1}{n}$. Therefore, in expectation, the
hunter will have correctly guessed the position that the rabbit will be at after
$n$ rounds. Since each round takes $n$ time steps the total number of expected
time steps is $\text{poly}(n)$.

\end{document}
