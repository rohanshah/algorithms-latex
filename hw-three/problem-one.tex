\documentclass[12pt]{article}
\usepackage{fullpage}
\usepackage{titlesec}
\usepackage{tikz}
\usepackage{amsfonts,amssymb}
\usepackage{amsmath}
\usepackage{comment}
\relpenalty=9999
\binoppenalty=9999

\begin{document}
\pagestyle{plain}
\titleformat{\subsection}[runin]
  {\normalfont\large\bfseries}{\thesubsection}{1em}{}
\titleformat{\subsubsection}[runin]
  {\normalfont\large\bfseries}{\thesubsubsection}{1em}{}

\section*{Problem 1.}
Consider the following protocol: Alice (and Bob) randomly selects $\sqrt{4n}$
primes from the first $2n$ primes, computes $F_p(A)$ (and $F_p(B)$ respectively),
for each prime, and sends the fingerprints and primes to Charlie.
Charlie compares $F_{p'}(A)$ and $F_{p'}(B)$ for some prime $p'$ that both
Alice and Bob share. If $F_{p'}(A) = F_{p'}(B)$ or if there does not exist some
prime $p'$ that both Alice and Bob share then Charlie does nothing and the
protocol is repeated, otherwise Charlie outputs NO and the protocol terminates.
The protocol is repeated $(\log{n})^{100}$ times after which Charlie outputs YES.
The total number of bits Alice (and Bob) send to Charlie is roughly
$\sqrt{n}(\log{n})^{O(1)}$ since each of the $\sqrt{4n}$ primes and fingerprints
are represented by $O(\log{n})$ bits and the protocol is repeated only
$(\log{n})^{100}$ times. Clearly, if $A = B$, Charlie always outputs YES
correctly. However, if $A \neq B$, Charlie may output YES if either,
$F_{p'}(A) = F_{p'}(B)$ in every iteration or if Alice and Bob never share
a prime. For the first, we are looking for
$$\textbf{Pr}[F_{p'}(A) = F_{p'}(B) \mid A \neq B]$$
Define $C = |A-B|$. The above occurs only if $C \neq 0$ and $F_{p'}(C) = 0$
i.e. $p'$ is a prime divisor of $C$. Since there are at most $n$ prime divisors
of $C$ and Alice (and Bob) randomly chose from $2n$ primes, the probability
that $p'$ is a prime divisor of $C$ is $\frac{n}{2n} = \frac{1}{2}$.
The protocol is repeated $(\log{n})^{100}$ times, so the probability this false
positive occurs during each iteration is $\frac{1}{2}^{(\log{n})^{100}}$. For
the second false postive case we show that the probability that Alice and Bob
never share a prime is low. Consider the probability that Bob does not randomly
select any of the same primes as Alice in a single iteration; Bob must have
chosen one of the $2n-\sqrt{4n} \choose \sqrt{4n}$ sets of primes that do not
include any of the $\sqrt{4n}$ primes Alice chose. The probability that this
occurs is
$$\frac{{2n-\sqrt{4n} \choose \sqrt{4n}}}{{2n \choose \sqrt{4n}}} << \frac{1}{2} $$
Again since the protocol is repeated $(\log{n})^{100}$ times the probability
that Alice and Bob never share a prime is much less than
$\frac{1}{2}^{(\log{n})^{100}}$. So the probability that Charlie outputs
correctly is $1-1/\text{poly}(n)$.
\end{document}
