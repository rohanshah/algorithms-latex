\documentclass[12pt]{article}
\usepackage{fullpage}
\usepackage{titlesec}
\usepackage{tikz}
\usepackage{amsfonts,amssymb}
\usepackage{amsmath}
\usepackage{comment}
\relpenalty=9999
\binoppenalty=9999

\begin{document}
\pagestyle{plain}
\titleformat{\subsection}[runin]
  {\normalfont\large\bfseries}{\thesubsection}{1em}{}
\titleformat{\subsubsection}[runin]
  {\normalfont\large\bfseries}{\thesubsubsection}{1em}{}

\section*{Problem 1.}
Consider the following protocol. Alice and Bob each select $\sqrt{4n}$ primes
at random from the first $2n$ primes i.e. from the range $[1...\tau]$ where
$\tau = 2n\log{n}$ which by the Prime Density Theorem has $\Theta(2n)$
primes. Both Alice and Bob compute $F_p(A)$ and $F_p(B)$, respectively, for
each of the $\sqrt{4n}$ primes they selected. Alice and Bob then send all of their
respective fingerprints and primes to Charlies. Note: The number of bits
each person sends to Charlie is $\sqrt{4n}\cdot O(\log{n})$. Proof: Since
$\tau = n^{O(1)}$ each prime and fingerprint can be represented by
$O(\log{n})$ bits. Finally, Charlie compares $F_{p'}(A)$ and $F_{p'}(B)$ for
some prime $p'$ that is shared by both Alice and Bob. If $F_{p'}(A) = F_{p'}(B)$
or if there does not exist some $p'$ that both Alice and Bob share then Charlie
does nothing and the protocol is repeated, otherwise Charlie outputs NO and the
protocol terminates. The protocol is repeated $(\log{n})^{O(1)}$ times after
which Charlie outputs YES. Note that the totalnumber of bits exchanced is at most
$(\log{n})^{O(1)} \cdot \sqrt{4n}\cdot O(\log{n}) = \sqrt{4n}(\log n)^{O(1)}$
as required. The analysis for this protocol consists of two parts:
the probability of a false positive and the probability that Alice and Bob share
a prime during each iteration of the protocol. For the first, we are looking for
$\textbf{Pr}[F_{p'}(A) = F_{p'}(B) \mid A \neq B]$. Define $C = |A-B|$. There
is a false positive if $C \neq 0$ but $F_{p'}(C) = 0$ i.e. $p'$ is a prime divisor
of $C$. There are at most $n$ prime divisors of $C$ and we are randomly chosing
from the first $2n$ primes so the probability that $p'$ is a prime divisor of
$C$ is $\frac{n}{2n} = \frac{1}{2}$. But since we repeat the protocol
$(\log{n})^{O(1)}$ times, the probability that a false positive occurs during
each repetition is $\frac{1}{2}^{(\log{n})^{O(1)}} = \frac{1}{n^{O(1)}}$. So
the probability that Charlie outputs correctly is $1-1/\text{poly}(n)$. Now
we show the probability that Alice and Bob share a prime during each iteration
is high. Consider the probability that Bob does not randomly select any of the
same primes as Alice in an iteration; Bob must choose from one of the
$2n-\sqrt{4n} \choose \sqrt{4n}$ sets that do not include any of the
$\sqrt{4n}$ primes Alice chose. The probability that this occurs is then
$\frac{{2n-\sqrt{4n} \choose \sqrt{4n}}}{{2n \choose \sqrt{4n}}}$
(which asymptotes around $.135$). So with high probility, Alice and Bob
choose at least one prime in common during each iteration of the protocol.
\end{document}
