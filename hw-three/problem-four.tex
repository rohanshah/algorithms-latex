\documentclass[12pt]{article}
\usepackage{fullpage}
\usepackage{titlesec}
\usepackage{tikz}
\usepackage{amsfonts,amssymb}
\usepackage{amsmath}
\usepackage{comment}
\relpenalty=9999
\binoppenalty=9999

\begin{document}
\pagestyle{plain}
\titleformat{\subsection}[runin]
  {\normalfont\large\bfseries}{\thesubsection}{1em}{}
\titleformat{\subsubsection}[runin]
  {\normalfont\large\bfseries}{\thesubsubsection}{1em}{}

\section*{Problem 4.}
Consider a directed path graph of length $k$ such that every vertex $v_i$ in the
path has an edge to vertices $v_0$ to $v_{i-1}$ and an edge to $v_{i+i}$. Then
we have $h_{i(i+1)} = i \cdot h_{(i-1)i}$ since the number of edges out of $v_i$
is $i+1$, $i$ of which move away from $v_{i+1}$, the edge $(i-1, i)$ will be
traveresed $i$ times in expectation before the edge $(i,i+1)$ is traversed. The
base case $h_{01} = 1$ by the definition of the graph. So inductively we have
$h_{0k} = k!$. Now consider an arbitrary graph $G$ and consider a vertex
$u$ as a starting vertex for a random walk. Partition the graph into two
sets $S$ and $T$ such that initially only $u$ is in $S$ and every other vertex
is in $T$. We perform random walks from $S$ to $T$ and once a vertex in $T$ is
reached we add it to $S$. There is always at least one edge from $S$ to $T$ since
the graph is strongly connected. Fix a path from $S$ to $T$. In the worst case
we need to traverse every vertex in $S$ before reaching $T$. Above we showed that
our hitting time for a directed path graph can be $k!$. We have covered $G$
only when every vertex is in $S$ and no vertex is in $T$. Given that $S$ increases
by one vertex at a time, in the worst case the cover time of $G$ is the hitting
time from $S$ to $T$ for each $n-1$ vertices in $T$ which is equal to
$$\sum_{k=1}^{n} k! < (n+1)!$$
Consider a lollipop graph where the candy is a clique of size $\frac{n}{2}$ and
the stem is a path graph such that each vertex in the path has an edge to every
node in the clique. Let $u$ be the vertex that connects the clique to the
path and $v$ be the edge at the end of the path. The hitting time $h_{uv}$ is
again defined as the product of each edge's hitting time in the path. Since
each edge in the path has an edge to every vertex in the clique, the hitting time
of taking an edge $(i,i+1)$ in the path is slightly more than $\frac{n}{2}$.
Since there are $\frac{n}{2}$ edges in the path,
$h_{uv} = \frac{n}{2}^{\frac{n}{2}}$ which is $n^{\Omega(n)}$.

\end{document}
