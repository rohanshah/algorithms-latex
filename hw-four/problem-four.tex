\documentclass[12pt]{article}
\usepackage{fullpage}
\usepackage{titlesec}
\usepackage{tikz}
\usepackage{amsfonts,amssymb}
\usepackage{amsmath}
\usepackage{comment}
\relpenalty=9999
\binoppenalty=9999

\begin{document}
\pagestyle{plain}
\titleformat{\subsection}[runin]
  {\normalfont\large\bfseries}{\thesubsection}{1em}{}
\titleformat{\subsubsection}[runin]
  {\normalfont\large\bfseries}{\thesubsubsection}{1em}{}

\section*{Problem 4a.}
Consider the following graph $G(V,E)$ such that $|V| = n$,
$V = \{u, v_1, ..., v_{n-1}\}$ and \\
$E = \{(u,v_i) \mid i \in [1..n-1]\}$. Conceptually, this graph has a center
vertex $u$ and $n-1$ outer vertices $v_i$ with edges only to $u$. Now consider
an implementation of the algorithm where, given an edge $(u,v_i)$, the algorithm
choses the vertex $v_i$ to be added to $C$. The algorithm outputs a cover
$C = \{v_1,...,v_{n-1}\}$ since each vertex $v_i$ only covers a single edge.
However, the smallest vertex cover is simply the set $C = \{u\}$ since the
vertex $u$ covers every edge in $G$. Therefore the set returned by the algorithm
is $\Omega(n)$ times larger than the smallest vertex cover.

\section*{Problem 4b.}
Let $C^*$ be the optimal vertex cover for some graph $G(V,E)$. Let
$S \subseteq E$ be the set of edges covered by some vertex $v^* \in C^*$.
Consider when the algorithm is looking at each edge $(v^*,w) \in S$. With equal
probability $1/2$ either $v^*$ or $w$ is added to $C$. Note that if $v^*$ is
added to $C$ all the edges in $S$ are covered and do not contribute any more
vertices to $C$. Since the vertices of an edge are added to $C$ with equal
probability $1/2$, in expectation only two edges need to be examined before
$v^*$ is added to $C$. Therefore, for each vertex $v^* \in C^*$, two vertices
are added to $C$, in expectation (since each edge contributes exactly one vertex
to $C$ and we need to examine two edges per $v^*$ before $v^*$ is added to $C$).
Since there are a total of $|C^*|$ such sets $S$ (one for each vertex
$v^* \in C^*$), a total of $2|C^*|$ vertices are added to $C$ before the
algorithm returns a vertex cover (one that contains all $v^* \in C^*$).
Therefore, the algorithm returns a 2-approximation to the optimal vertex cover
in expectation.

\end{document}
