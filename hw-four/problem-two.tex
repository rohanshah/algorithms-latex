\documentclass[12pt]{article}
\usepackage{fullpage}
\usepackage{titlesec}
\usepackage{tikz}
\usepackage{amsfonts,amssymb}
\usepackage{amsmath}
\usepackage{comment}
\relpenalty=9999
\binoppenalty=9999

\begin{document}
\pagestyle{plain}
\titleformat{\subsection}[runin]
  {\normalfont\large\bfseries}{\thesubsection}{1em}{}
\titleformat{\subsubsection}[runin]
  {\normalfont\large\bfseries}{\thesubsubsection}{1em}{}

\section*{Problem 2a.}
Let $0.b_1b_2...$ be the binary fractional representation of $p_1$. Let $X_i$
be a independent binary random variable that represents the outcome of a coin
toss ($X_i = 1$ if heads, $X_i = 0$ if tails). Consider the following algorithm:
for $i = 1,2,...$ if $b_i > X_i$ return $a_1$, if $b_i < X_i$ return $a_2$,
otherwise (if $b_i == X_i$) loop. Let $X = 0.X_1X_2...$ This algorithm returns
$a_1$ when $X < p_1$ and returns $a_2$ when $X > p_1$. Therefore,
$\textbf{Pr}[X < p_1] = p_1$ and $\textbf{Pr}[X > p_1] = 1 - p_1 = p_2$.
Thus the algorithm correctly outputs $a_1$ and $a_2$ with their associated
probabilities. Since the algorithm terminates when $b_i \neq X_i$ and loops
whenever $b_i = X_i$, the algorithm terminates with probability $1/2$ during
each iteration. In expectation, the algorithm uses $2$ coin flips. By Markov's
inequality, the algorithm uses more than $200$ coin flips with probability at
most  $\frac{1}{100}$. Therefore, with high probability the algorithm requires
$O(1)$ coin tosses to return either $a_1$ or $a_2$. If after $200$ coin tosses
the algorithm does not return an element we can stop and run it again.

\section*{Problem 2b.}
Given a set $S = \{a_1,a_2,...,a_n\}$ with associated probabilities $p_i$ we can
generate $n$ ranges of the form
$[\sum^{i-1}_{j=1} p_j, \sum^i_{j=1} p_j)$ i.e.
$[0,p_1), [p_1, p_1 + p_2),..., [\sum^{n-1}_{i=1} p_i, 1)$.
Note that each range has length equal to
$$\sum^i_{j=1} p_j - \sum^{i-1}_{j=1} p_j = p_i$$
The probability of generating a random number from the interval
$[0,1)$ that falls in the $i$-th range is $p_i$. Since each range is strictly
greater than and less than its neighbors we can create a balanced binary search tree
with $n$ nodes such that each node contains an element $a_i$ and each node has
an associated key defined by the $i$-th range. To choose an element $a_i$
with probability $p_i$ we do the following. As above we generate a random number
$X = 0.X_1X_2...$ using the fair coin. Each time $X_i = 0$ we can lower bound
the range that $X$ falls in. Similarly each time $X_i = 1$ we can upper bound
the range that $X$ falls in. Starting at the root, we generate $X$. If the
upper and lower bound of $X$ fall within the range of the current node
we return the node's element $a_i$. Otherwise, if $X$'s lower bound is greater
than the range of the current node we move to the nodes right child. Similarly,
if $X$'s upper bound is less than the current node's range we move to the left
child. This algorithm continue generating $X$ until an element $a_i$ has been
returned. The correctness of this algorithm (briefly described above) is that
generating a random number in $[0,1)$ has probability $p_i$ of falling in the
$i$-th range and therefore using this algorithm there is exactly a $p_i$
probability of chosing an element $a_i$ from the tree. Note that each coin flip
reduces the range of $X$ by a half. Further, the tree has $n$ nodes and is a
balanced binary search tree so it has a height of $log n$. From part (a) we
know that only $O(1)$ coin flips (in expectation) are necessary to determine
whether $X$ is greater than or less a number of the form $0.b_1b_2...$. Each
range is defined by exactly two such numbers. Therefore, using $O(1)$ coin flips
we can determine which side of the range $X$ falls for each node. Since the tree
has height $\log n$ only $O(\log n)$ coin flips are needed in expectation
before an element $a_i$ is returned.

\end{document}
