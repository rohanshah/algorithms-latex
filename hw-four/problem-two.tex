\documentclass[12pt]{article}
\usepackage{fullpage}
\usepackage{titlesec}
\usepackage{tikz}
\usepackage{amsfonts,amssymb}
\usepackage{amsmath}
\usepackage{comment}
\relpenalty=9999
\binoppenalty=9999

\begin{document}
\pagestyle{plain}
\titleformat{\subsection}[runin]
  {\normalfont\large\bfseries}{\thesubsection}{1em}{}
\titleformat{\subsubsection}[runin]
  {\normalfont\large\bfseries}{\thesubsubsection}{1em}{}

\section*{Problem 2a.}
Let $0.b_1b_2...$ be the binary fractional representation of $p_1$. Let $X_i$
be a independent binary random variable that represents the outcome of a coin
toss ($X_i = 1$ if heads, $X_i = 0$ if tails). Consider the following algorithm:
for $i = 1,2,...$ if $b_i = 1$ and $X_i = 0$ then return $a_1$, if $b_1 = 0$ and
$X_i = 1$ return $a_2$, otherwise (if $b_i = X_i$) then increment $i$ and repeat.
Let $X = 0.X_1X_2...$. The above algorithm returns $a_1$ when $X < p_1$ and
returns $a_2$ when $X > p_1$. Note that since each $X_i$ is generated
uniformly at random that $X$ is a number generated uniformly at random from
the interval $[0,1]$. Therefore we have that $\textbf{Pr}[X < p_1] = p_1$ and
$\textbf{Pr}[X > p_1] = 1 - p_1 = p_2$. Thus the algorithm correctly outputs
$a_1$ and $a_2$ with their associated probabilities. Since the algorithm
terminates when $b_i \neq X_i$ and loops whenever $b_i = X_i$, the algorithm
terminates with probability $1/2$ during each iteration. Therefore, in
expectation, the algorithm requires $2$ coin flips to terminate.

\section*{Problem 2b.}
Given a set $S = \{a_1,a_2,...,a_n\}$ with associated probabilities $p_i$ we can
generate $n$ ranges of the form
$[\sum^{i-1}_{j=1} p_j, \sum^i_{j=1} p_j)$ i.e.
$[0,p_1), [p_1, p_1 + p_2),..., [\sum^{n-1}_{i=1} p_i, 1)$. Note that each key
has length equal to
$$\sum^i_{j=1} p_j - \sum^{i-1}_{j=1} p_j = p_i$$
Therefore the probability of generating a random number from the interval
$[0,1)$ that falls in the $i$-th range is $p_i$. Since each range is strictly
greater than and less than its neighbors we can create a balanced binary search tree
with $n$ nodes such that each node contains an element $a_i$ and each node has
an associated key defined by the $i$-th range. Now to choose an element $a_i$
with probability $p_i$ we do the following. As above we generate a random number
$X = 0.X_1X_2...$ using the fair coin. Each time $X_i = 0$ we can lower bound
the range that $X$ falls in. Similarly each time $X_i = 1$ we can upper bound
the range that $X$ falls in. Starting at the root, we generate $X$ until either
the upper and lower bound of $X$ fall within the range of the current node
at which point we return the element $a_i$ at the node. Otherwise if $X$'s lower
bound is greater than the range of the current node we move to the nodes right
child and similarly if $X$'s upper bound is less than the current node's range
we move to the left child and continue generating $X$ until an element $a_i$
has been returned. The correctness of this algorithm (briefly described above)
is that generating a random number in $[0,1)$ has probability $p_i$ of falling
in the $i$-th range and therefore using this algorithm there is exactly a $p_i$
probability of chosing an element $a_i$ from the tree. Note that each coin flip
reduces the range of $X$ by a half. Also note that the tree has $n$ nodes and is
a balanced binary search tree so it has a height of $log n$. Therefore only
$O(\log n)$ coin flips are needed in expectation before an element $a_i$ is
returned

\end{document}
