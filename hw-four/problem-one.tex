\documentclass[12pt]{article}
\usepackage{fullpage}
\usepackage{titlesec}
\usepackage{tikz}
\usepackage{amsfonts,amssymb}
\usepackage{amsmath}
\usepackage{comment}
\relpenalty=9999
\binoppenalty=9999

\begin{document}
\pagestyle{plain}
\titleformat{\subsection}[runin]
  {\normalfont\large\bfseries}{\thesubsection}{1em}{}
\titleformat{\subsubsection}[runin]
  {\normalfont\large\bfseries}{\thesubsubsection}{1em}{}

\section*{Problem 1a.}
Let $C$ be the collection of sets $S_1, S_2, ...,S_m$ and let $x_i$ be a
variable such that $x_i = 1$ if the element $i \in [1..n]$ is in $H$ and
$x_i = 0$ otherwise. Then an integer linear program to find an optimal solution
to this problem can be stated as:
$$ \textbf{minimize } \sum_{i=1}^n x_i $$
$$ \textbf{ subject to } $$
$$ \forall S_j \in C,\ \sum_{x_i \in S_j} x_i \ge 1 $$
$$ \forall i \in [1..n],\ x_i \in \{0,1\} $$

\section*{Problem 1b.}
Consider the following randomized polytime algorithm. Let $x^*_i$ be the value
associated with each $x_i$ in the optimal linear programming solution. For each
$i \in [1..n]$, assign $i$ to the set $T$ with probability $x^*_i$. Repeat this
procedure $10$ times and return $T$. The expected size of $T$ after a single
iteration is $\sum^n_{i=1} x^*_i = F^*$ so, in expecation, after $10$ iterations
the size of $T$ is at most $10 F^*$ (if during each iteration a disjoint set of
$i \in [1..n]$ were selected to be in $T$).

\section*{Problem 1c.}



\end{document}
