\documentclass[12pt]{article}
\usepackage{fullpage}
\usepackage{titlesec}
\usepackage{tikz}
\usepackage{amsfonts,amssymb}
\usepackage{amsmath}
\usepackage{comment}
\relpenalty=9999
\binoppenalty=9999

\begin{document}
\pagestyle{plain}
\titleformat{\subsection}[runin]
  {\normalfont\large\bfseries}{\thesubsection}{1em}{}
\titleformat{\subsubsection}[runin]
  {\normalfont\large\bfseries}{\thesubsubsection}{1em}{}

\section*{Problem 1a.}
Let $C$ be the collection of sets $S_1, S_2, ...,S_m$ and let $x_i$ be a
variable such that $x_i = 1$ if the element $i \in [1..n]$ is in $H$ and
$x_i = 0$ otherwise. Then an integer linear program to find an optimal solution
to this problem can be stated as:
$$ \textbf{minimize } \sum_{i=1}^n x_i $$
$$ \textbf{ subject to } $$
$$ \forall S \in C,\ \sum_{i \in S} x_i \ge 1 $$
$$ \forall i \in [1..n],\ x_i \in \{0,1\} $$

\section*{Problem 1b.}
Consider the following randomized polytime algorithm. Let $x^*_i$ be the value
associated with each $x_i$ in the optimal linear programming solution. For each
$i \in [1..n]$, assign $i$ to the set $T$ with probability $x^*_i$.
\vspace{2mm}
\newline
\textbf{Lemma:} $\textbf{E}[|T|] = F^*$ i.e. the size of the set returned
by the rounding scheme is equal to the optimal value of a solution to the linear
programming relaxation, in expectation.
\vspace{2mm}
\newline
\textbf{Proof:} Let $X_i$ be a 0/1 random variable such that
$X_i = 1$ if $x_i \in T$ and $X_i = 0$ otherwise.
$$\textbf{E}[X_i] = \textbf{Pr}[X_i = 1] = x^*_i$$
Let $X = X_1 + ... + X_n$. By linearity of expectation we have
$\textbf{E}[X] = \textbf{E}[X_1] + ... + \textbf{E}[X_n]$.
$$\textbf{E}[X] = \sum^n_{i=1} x^*_i = F^*$$
Now consider the probability that $|T\cap S_i|\ge 1$. First we have that
$\textbf{Pr}[|T\cap S_i| \ge 1] = 1 - \textbf{Pr}[|T\cap S_i| = 0]$.
Without loss of generality let $S_i = \{1,...,k\}$.
$$ \textbf{Pr}[|T\cap S_i| = 0] =
\prod_{i=1}^k 1-x^*_i \le \prod_{i=1}^k e^{-x^*_i}
= e^{-\sum^k_{i=1} x^*_i} \le \frac{1}{e}$$
The inequalities above come from properties of the linear program i.e. that
$x^*_i \in [0,1]$ and $\sum^k_{i=1} x^*_i \ge 1$. So $|S_i \cap T| \ge 1$ with
probability at least $1-\frac{1}{e}$.
\vspace{3mm}
\newline
We can repeat the above algorithm $10$ times (independently) and take the union
of the sets it returns. From the lemma above, the expected size of our final set
is at most $10F^*$. Further, the probability that $|S_i \cap T| = 0$ after
$10$ independent iterations less than $\left(\frac{1}{e}\right)^{10}$.
Therefore,
$$\textbf{Pr}[|S_i\cap T| \ge 1] \ge 1 - \left(\frac{1}{e}\right)^{10}$$
Let $X_i$ be a 0/1 random variable such that $X_i = 1$ if $|S_i \cap T| \ge 1$
and $X_i = 0$ otherwise.
$$\textbf{E}[X_i] = \textbf{Pr}[X_i = 1] = 1 - \left(\frac{1}{e}\right)^{10}$$
Let $X = X_1 + ... + X_m$. By linearity of expectation we have
$\textbf{E}[X] = \textbf{E}[X_1] + ... + \textbf{E}[X_n]$.
$$\textbf{E}[X] = \sum^m_{i=1} 1 - \left(\frac{1}{e}\right)^{10} > \frac{m}{10}$$

\section*{Problem 1c.}
If we instead repeat the above algorithm $10 \cdot \log m$ times then the
expected size of $T$ is $O(\log m)\cdot F^*$. Further we have that
$$\textbf{Pr}[|S_i\cap T| \ge 1] = 1 - \textbf{Pr}[|S_i\cap T| = 0]
= 1 - \left(\frac{1}{e}\right)^{10\cdot \log m} = 1 - \frac{1}{m^{10}}$$
Therefore, $|S_i\cap T| \ge 1$ for $1 \le i \le m$ with probability $1-o(1)$ 

\end{document}
