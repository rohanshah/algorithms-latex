\documentclass[12pt]{article}
\usepackage{fullpage}
\usepackage{titlesec}
\usepackage{tikz}
\usepackage{amsfonts,amssymb}
\usepackage{amsmath}
\usepackage{comment}
\relpenalty=9999
\binoppenalty=9999

\begin{document}
\pagestyle{plain}
\titleformat{\subsection}[runin]
  {\normalfont\large\bfseries}{\thesubsection}{1em}{}
\titleformat{\subsubsection}[runin]
  {\normalfont\large\bfseries}{\thesubsubsection}{1em}{}

\section*{Problem 3a.}
Without loss of generality let each clause in $\Phi$ have exactly $3$
distinct literals. Let $S$ be a set of $C$ disjoint clauses in $\Phi$.
A single clause in $\Phi$ (and $S$ respectively) can be satisfied by $7$ of the
$8$ possible assingments (since there is only one assignment that does
not satisfy the clause i.e. the one where each of the $3$ literals evaluates to
false). Since the clauses in $S$ are disjoint, an assignment for the variables
in one clause does not affect the satisfiabilty of another clause. Therefore,
there are exactly $7^C$ satisfying assignments out of the total $8^C$ possible
assignments for the set $S$. For $\Phi$ to be satisfied, both the variables in
$S$ and the variables not in $S$ need to have a satisfying assignment. Now
consider the remaining $n-3C$ variables not in $S$. At best, all the $2^{n-3C}$
possible assignments satisfy the clauses not in $S$. Combining all the possible
satisfying assignments for the two sets there are at most $7^C \cdot 2^{n-3C}$
possible satisfying assignments for $\Phi$ out of the total $2^n$ possible
assignments. To satisfy the condition that at least $1\%$ of the possible
assignments satisfy all the clauses in the formula $\Phi$ we have the following
inequality:
$$\frac{7^C \cdot 2^{n-3C}}{2^n} \ge \frac{1}{100} $$
which gives a threshold of $34$, after which the inequality does not hold
and the fact that $1\%$ of the assignments satify $\Phi$ is contradicted.

\section*{Problem 3b.}
Consider the following algorithm for finding a set $V$ of at most $3C$ variables
with the property that every clause in $\Phi$ has at least one variable in $V$.
Maintain a list of the clauses each variable belongs to. Chose the variable that
belongs to the most number of clauses to be in $V$. Repeat this process modulo
the clauses that are already covered by selected variables until every clause
has at least one variable in $V$. This algorithm takes $O(m)$ preprocessing time
to create the list of variable-clause relationships and requires $O(n)$ space.
The algorithm runs in at most $O(n)$ (to remove the newly covered clauses from
the variable-clause relationship list) since the algorithm iterates at most $3C$
times (which is yet to be shown). Finally we can show that the size of $V$ is at
most $3C$. Consider the largest set $S$ of disjoint clauses in $\Phi$. From
part (a) we know that $|S| < C$ therefore $S$ contains at most $3C$ variables.
Since $S$ is the largest set of disoint clauses, every clause in $\Phi$ has at
least one variable in $S$. Proof by contradiction. If there exists a clause with
no variables in $S$ then that clause can be added to $S$ since it is disjoint
from every other clause in $S$ by definition. But that makes the size of
$S$ equal to $|S|+1$ therefore the orignal set $S$ was not the largest set of
disjoint clauses, a contradiction. Now consider the subset of all the variables
in $S$ that cover every clause in $\Phi$. Clearly each of these variables covers
enough clauses in $\Phi$ such that all of $\Phi$ is covered by them. Since the
algoirithm described above choses variables that cover the maximum number of
clauses to cover $\Phi$, the algorithm needs to select, at most, as many variables
as the subset just described, which has at most $3C$ variables. The correctness
of the algorithm is straightforward since it runs until every clause in $\Phi$
is covered by at least one variable in $V$.

\section*{Problem 3c.}



\end{document}
