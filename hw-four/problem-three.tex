\documentclass[12pt]{article}
\usepackage{fullpage}
\usepackage{titlesec}
\usepackage{tikz}
\usepackage{amsfonts,amssymb}
\usepackage{amsmath}
\usepackage{comment}
\relpenalty=9999
\binoppenalty=9999

\begin{document}
\pagestyle{plain}
\titleformat{\subsection}[runin]
  {\normalfont\large\bfseries}{\thesubsection}{1em}{}
\titleformat{\subsubsection}[runin]
  {\normalfont\large\bfseries}{\thesubsubsection}{1em}{}

\section*{Problem 3a.}
Without loss of generality let each clause in $\Phi$ have exactly $3$
distinct literals. Let $S$ be a set of $C$ disjoint clauses in $\Phi$.
A single clause can be satisfied by exactly $7$ out of $8$ possible assingments
(since there is only one assignment that does not satisfy the clause i.e. the
one where each of the $3$ literals evaluates to false). Since the clauses in $S$
are disjoint, an assignment for the variables in one clause does not affect the
satisfiabilty of another clause. Therefore, there are exactly $7^C$ satisfying
assignments out of the total $8^C$ possible assignments for the set $S$. For
$\Phi$ to be satisfied, the variables in $S$ and the variables not in $S$ need
to have a combined satisfying assignment. Now consider the remaining $n-3C$
variables not in $S$. At best, all the $2^{n-3C}$ possible assignments satisfy
the clauses not in $S$. Combining all the possible satisfying assignments for
the two sets of clauses there are at most $7^C \cdot 2^{n-3C}$ possible
satisfying assignments for $\Phi$ out of the total $2^n$ possible assignments.
To satisfy the condition that at least $1\%$ of the possible assignments satisfy
all the clauses in the formula $\Phi$ we have the following inequality:
$$\frac{7^C \cdot 2^{n-3C}}{2^n} \ge \frac{1}{100} $$
which gives a constant upper bound of $34$, after which the inequality does not
hold and the fact that $1\%$ of the assignments satify $\Phi$ is contradicted.

\section*{Problem 3b.}
Consider the following algorithm for finding a set $V$ of at most $3C$ variables
with the property that every clause in $\Phi$ has at least one variable in $V$.
Maintain a list of the clauses each variable belongs to. Choose the variable
that belongs to the most number of clauses to be in $V$. Repeat this process,
modulo the clauses that are already covered by selected variables, until every
clause has at least one variable in $V$. This algorithm takes $O(m)$
preprocessing time to create the list of variable-clause relationships. Since
the algorithm iterates at most $3C$ times (which is yet to be shown), the
algorithm runs in at most $O(n)$ time (to remove the newly covered clauses from
the variable-clause relationships). We can show that the size of $V$ is at most
$3C$. Consider the largest set $S$ of disjoint clauses in $\Phi$. From part (a)
we know that $|S| \le C$ therefore $S$ contains at most $3C$ variables. Since
$S$ is the largest set of disoint clauses, every clause in $\Phi$ has at least
one variable in $S$. Proof by contradiction: if there exists a clause with no
variables in $S$ then that clause can be added to $S$ since it is disjoint from
every other clause in $S$, which increases the size of $S$ to $|S|+1$ therefore
$S$ was not the largest set of disjoint clauses, a contradiction. Now consider a
set $S' \subseteq S$ of all the variables in $S$ that cover every clause in
$\Phi$. Each of these variables covers some number of clauses in $\Phi$.
Since the above algorithm choses variables that cover the maximum number of
clauses to cover $\Phi$, the algorithm would select, at most, all the variables
in $S'$ which is of size at most $3C$. Otherwise it would chose other variables
in lieu of variables in $S'$ that cover more clauses, thereby needing less total
variables to cover $\Phi$. Therefore, the algorithm produces a set of at most
$3C$ variables such that every clause in $\Phi$ is covered by at least one
of the variables.

\section*{Problem 3c.}
Using part (b) we can construct a set $V$ of at most $3C$ variables with the
property that every clause in $\Phi$ has one variable in $V$. Now consider
chosing one of the $2^{3C}$ possible assignments for the variables in $V$. We
can use this assignment to reduce $\Phi$ in the following way. First, we can
remove the clauses that are satisfied by the assignment. Second, we can reduce
each of the unsatisfied clauses to a 2-CNF clause by removing the variable from
$V$ who's current assignment does not satisfy it. Now we are left with an 2-SAT
instance (note this instance does not contain any variables from $V$). We can
solve this 2-SAT instance in polynomial time. If that algorithm finds a
satisfying assignment then we can combine it with the assignment for the $3C$
variables in $V$ to get a satisfying assignment for $\Phi$. Otherwise, we repeat
this procedure, chosing a new assignment for the variables in $V$. As noted
above, this algorithm will iterate at most $2^{3C}$ times, once for each
possible assignment for the variables in $V$. Since at least $1\%$ of the
possible asssignments satisfy $\Phi$ there is at least one assignment for the
variables in $V$ that satisfy $\Phi$ therefore the algorithm always returns a
satisfying assignment. The algorithm needs to solve at most $2^{3C}$ 2-SAT
instances. We can reduce $\Phi$ in $O(m)$ time and solve a 2-SAT intance in
polynomial time therefore the entire algorithm runs in polynomial time including
the preprocessing construction of the set $V$ from part (b).

\end{document}
