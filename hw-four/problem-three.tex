\documentclass[12pt]{article}
\usepackage{fullpage}
\usepackage{titlesec}
\usepackage{tikz}
\usepackage{amsfonts,amssymb}
\usepackage{amsmath}
\usepackage{comment}
\relpenalty=9999
\binoppenalty=9999

\begin{document}
\pagestyle{plain}
\titleformat{\subsection}[runin]
  {\normalfont\large\bfseries}{\thesubsection}{1em}{}
\titleformat{\subsubsection}[runin]
  {\normalfont\large\bfseries}{\thesubsubsection}{1em}{}

\section*{Problem 3a.}
Without loss of generality let each clause in $\Phi$ have exactly $3$
distinct literals. Let $S$ be a set of $C$ disjoint clauses in $\Phi$.
A single clause in $\Phi$ (and $S$ respectively) can be satisfied by $7$ of the
$8$ possible assingments (since there is only one assignment that does
not satisfy the clause i.e. the one where each of the $3$ literals evaluates to
false). Since the clauses in $S$ are disjoint, an assignment for the variables
in one clause does not affect the satisfiabilty of another clause. Therefore,
there are exactly $7^C$ satisfying assignments out of the total $8^C$ possible
assignments for the set $S$. For $\Phi$ to be satisfied, both the variables in
$S$ and the variables not in $S$ need to have a satisfying assignment. Now
consider the remaining $n-3C$ variables not in $S$. At best, all the $2^{n-3C}$
possible assignments satisfy the clauses not in $S$. Combining all the possible
satisfying assignments for the two sets there are at most $7^C \cdot 2^{n-3C}$
possible satisfying assignments for $\Phi$ out of the total $2^n$ possible
assignments. To satisfy the condition that at least $1\%$ of the possible
assignments satisfy all the clauses in the formula $\Phi$ we have the following
inequality:
$$\frac{7^C \cdot 2^{n-3C}}{2^n} \ge \frac{1}{100} $$
which gives a threshold of $34$, after which the inequality does not hold
and the fact that $1\%$ of the assignments satify $\Phi$ is contradicted.

\section*{Problem 3b.}
Using our bound for $C$ from part (a), let $S$ be a set of at most $C$ disjoint
clauses in $\Phi$. By definition $S$ has at most  $3C$ variables. Further, if
$S$ is the largest set of disjoint clauses in $\Phi$ then every clause not in
$\Phi$ has a variable in $S$. Proof by contradiction: Say there exists a clause
in $\Phi$ that has no variables in $S$. Then that clause can be added to $S$
since it is disjoint from all the other clauses in $S$. Therefore $S$ was not
the largest disjoint set of clauses to begin with, a contradiction. Thus, an
equivalent problem to finding a set $V$ of at most $3C$ variables that have the
property that every clause in $\Phi$ has at least one variable in $V$, is to
find the largest set of disjoint clauses in $\Phi$ and return all the variables
it contains.


\section*{Problem 3c.}



\end{document}
