\documentclass[12pt]{article}
\usepackage{fullpage}
\usepackage{titlesec}
\usepackage{tikz}
\usepackage{amsfonts,amssymb}
\usepackage{amsmath}
\usepackage{comment}
\relpenalty=9999
\binoppenalty=9999

\begin{document}
\pagestyle{plain}
\titleformat{\subsection}[runin]
  {\normalfont\large\bfseries}{\thesubsection}{1em}{}
\titleformat{\subsubsection}[runin]
  {\normalfont\large\bfseries}{\thesubsubsection}{1em}{}

\section*{Problem 4a.}
The conditions $x_{i,l} + x_{i,r} + x_{i,b} + x_{i,t} \ge 1$
and $x_{i,l}, x_{i,r}, x_{i,b}, x_{i,t} \in \{0,1\}$ ensure that
$\forall 1 \le i \le n$ at least one $x_{i,\lambda} = 1$ which means that at
least one side of $r_i$ is extended to an edge, so the integer linear program
will always find a feasible solution to the problem. For each cell $c \in G$ the
set $P_c$ defines all the rectangles who have the potential to increase the
density of each point within $c$. Minimizing the number of such rectangles that
affect a cell once it is determined which side(s) of the rectangle are extended
is equivalent to minimizing the density for each cell. Minimizing this value for
every cell in $G$ then is equivalent to finding the minimal maximum density for
every point which is exactly the problem.

\section*{Problem 4b.}
Consider the following algorithm. Relax the above integer linear program for the
problem such that we change the last condition to
$$0\le x_{i,l}, x_{i,r}, x_{i,b}, x_{i,t} \le 1\ \ \ \ \ \forall 1 \le i \le n$$
Compute the optimal solution to the linear-program relaxation. Then perform the
following deterministic rounding scheme: for each $1 \le i \le n$, set the
$x_{i,\lambda}$ with the greatest value equal to $1$ and $0$ otherwise. This
rounding scheme always computes a feasible solution for the problem since for
every rectangle $r_i$ at least one direction $\lambda$ such that
$x_{i,\lambda} = 1$ is chosen for the rectangle to be extended to. The value
for each $x_{i,\lambda}$ in the rounded solution is at most $4$ times greater
than its value in the linear-program relaxation solution. This is true since
for $x_{i,\lambda}$ to be rounded to $1$ it must be the greatest value in the
linear-program solution which means that its value must have been greater than
or equal to $\frac{1}{4}$ in order for the condition
$x_{i,l} + x_{i,r} + x_{i,b} + x_{i,t} \ge 1$ to hold. In the LP solution there
exists some $c$ such that $\sum_{(i,\lambda) \in P_c} x_{i,\lambda} = Z$ i.e.
the optimal solution returned by the linear-program relaxation. In the worst
case, every $x_{i,\lambda}$ in the sumation is rounded to $1$. Since each
$x_{i,\lambda}$ is a most a factor of $4$ from its LP value and since the
LP solution is a bound on the OPT solution for the integer linear program, this
algorithm produces a $4$-approximate solution.

\section*{Problem 4c.}
Consider the following rounding scheme for the linear programming relaxation of
the given integer linear program formulation for the problem. For each
rectangle $r_i$ set one of the four $x_{i,\lambda} = 1$ in the integer solution
with probability equal to $x_{i,\lambda}$ i.e. their values in the LP solution.
This produces a feasible solution for the problem because one side, namely
$\lambda$ such that $x_{i,\lambda} = 1$, is chosen for the rectangle to be
extended to.


\end{document}
