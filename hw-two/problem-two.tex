\documentclass[12pt]{article}
\usepackage{fullpage}
\usepackage{titlesec}
\usepackage{tikz}
\usepackage{amsfonts,amssymb}
\usepackage{amsmath}
\usepackage{comment}
\relpenalty=9999
\binoppenalty=9999

\begin{document}
\pagestyle{plain}
\titleformat{\subsection}[runin]
  {\normalfont\large\bfseries}{\thesubsection}{1em}{}
\titleformat{\subsubsection}[runin]
  {\normalfont\large\bfseries}{\thesubsubsection}{1em}{}

\section*{Problem 2a.}
Let $X = X_1 + X_2 + \cdots + X_m$ where $m$ is the number of edges in $G$ and
$X_i$ is a binary random variable such that $X_i = 1$ if an edge $i$ is in the
cut and $X_i = 0$ otherwise. Let $i = (u,v)$ for some $u,v \in V$. Then
$$\textbf{Pr}[X_i = 1] = \textbf{Pr}[u \in S \text{ and } v \in T] \text{ or }
\textbf{Pr}[v \in S \text{ and } u \in T]$$
Since the algorithm assigns vertices to $S$ and $T$ independently and with equal
probability
$$\textbf{Pr}[X_i = 1] = \textbf{Pr}[u \in S]\cdot\textbf{Pr}[v \in T] +
\textbf{Pr}[v \in S]\cdot\textbf{Pr}[u \in T] =
\frac{1}{2}\cdot\frac{1}{2} + \frac{1}{2}\cdot\frac{1}{2} = \frac{1}{2}$$
Since $X_i$ are independent binary random variables
$$\textbf{E}[X_i] = \textbf{Pr}[X_i = 1] = \frac{1}{2}$$
And by the linearity of expectation
$$\textbf{E}[X] = \sum_{i=1}^m \textbf{E}[X_i] = \frac{m}{2}$$
Since the upper bound on the optimum solution for MAX-CUT is $m$ this is a
2-approximation algorithm in expectation.

\section*{Problem 2b.}
In part (a) each vertex $v$ is placed in $S$ and $T$ independently. We can
require something weaker, for each vertex to be placed pairwise-independently,
that is, for any pair of vertices $(u,v)$, $u$ and $v$ are placed in $S$ and $T$
independently. To show this, consider two edges $i \neq j$ such that
$i = (u,v)$ and $j = (u,v')$ share an endpoint in $G$. By the definition of
pairwise-independence, $u$ and $v$ are placed in $S$ and $T$ independently and
$u$ and $v'$ are also placed in $S$ and $T$ independently. Let $C$ be the cut
produced by the algorithm. Then $\textbf{Pr}[i \in C] = \textbf{Pr}[j \in C] =
\frac{1}{2}$ as shown above. In homework one, problem three we showed that $n$
pairwise independent random variables can be constructed using
$\lfloor{\log n}\rfloor + 1$ random bits. Consider the following algorithm:
Assign each of the $n$ constructed pairwise independent random variables,
$Y_1...Y_n$, to the each of the vertices $v_1...v_n \in V$. Run the algorithm
from part (a), i.e. place $v_i$ in $S$ if $Y_i = 1$ and in $T$ otherwise.
We again have a 2-approximate randomized algorithm in expectation for MAX-CUT
by the analysis above.

\section*{Problem 2c.}
Consider the algorithm from part (b) as a black box that when fed an
assignment for the ($\log n$) random bits produces a 2-approximate solution for
a MAX-CUT problem in expectation. If we enumerate all possible assignments to
the $\log n$ random bits we can generate $2^{\log n} = n$ solutions for MAX-CUT.
Since the algorithm produces a 2-approximate solution in expectation if we
enumerate all possible assignments then we are guarenteed to be given at least
one 2-approximate solution for the problem. We can then return the largest
solution output by the algorithm after all $n$ runs as our answer to obtain a
deterministic 2-approximation algorithm.

\end{document}
